\chapter{Introduction}

\section{Background and Motivation}
% Explain the significance of document understanding in financial accounting. 
% Discuss the limitations of traditional rule-based and machine learning approaches.
% Highlight the potential of generative AI to revolutionize document understanding tasks.

Most content is intended for human readers. However, because of this a computer can only very limitedly process this content. As a consequence, for automation purposes in the first step the given content has to be converted in a format that a machine can read.

\section{Research Objectives}
%Clearly define the specific goals of the thesis, such as:
%Developing a generative AI model for extracting structured data from financial documents.
%Evaluating the accuracy and efficiency of the model compared to existing methods.
%Exploring the potential applications of the model in real-world financial scenarios.
%Wie effektiv können mit Hilfe von (generativen) Ki Modellen relevante Daten aus semi-strukturierten Finanzdokumenten extrahiert werden am Beispiel der Rechnungsverarbeitung mit dem Document Understanding Modus von UiPath?

%How effectively can AI models extract relevant data from semi-structured financial documents?

%This will be exemplified by invoice processing with UiPath's Document Understanding module.


The research objectives of this thesis include the presentation of the technical foundations of Document Understanding systems and their application on a theoretical basis. For this, the thesis is structured into a theoretical and an experimental part.

In the theoretical part, the common Document Understanding techniques \ac{OCR}, \ac{IE}, and \ac{NER} are introduced. 
Furthermore, it covers central concepts of \ac{NLP} and \ac{ML} with a focus on \acp{ANN} for Document Understanding. 
Finally, the potential of Document Understanding for the automation of business processes is presented.

After the theoretical foundation is set, this thesis deals with a concrete practical example. The aim is to find evidence for how effectively AI models can extract relevant data from semi-structured financial documents. This will be exemplified by using the UiPath Document Understanding software to extract data from unseen invoice documents.
For this purpose, an invoice processing system is developed using UiPath Document Understanding and tested with a set of actual invoices. Furthermore, the results of the pre-trained out-of-the-box model provided by UiPath are compared to a custom trained model.  
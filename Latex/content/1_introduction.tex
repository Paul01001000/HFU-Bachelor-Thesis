\chapter{Introduction}

\section{Background and Motivation}
% Explain the significance of document understanding in financial accounting. 
% Discuss the limitations of traditional rule-based and machine learning approaches.
% Highlight the potential of generative AI to revolutionize document understanding tasks.

Recent advancements in Artificial Intelligence (AI), particularly in the areas Natural Language Processing (NLP) and Computer Vision, provide new innovative approaches how businesses interact with information. Documents whose content and layout is intended for human readers can now be processed by intelligent computer systems as well. This offers great opportunities to automate tedious and time-consuming document-based process steps.

A potential place of use for AI-based Document Understanding is the automation of invoice processing.
Traditional manual invoice processing is time-consuming and inefficient, because it involves manual data entry and verification. This consumes valuable employee time and resources. The use of AI-based Document Understanding for this process is the main topic of interest of this thesis.

Because of its diverse application potential, Document Understanding technology has been of interest for researches for several decades. Over time, Document Understanding has been based on several technological approaches and researchers always try to utilize the most recent technological advancements to improve and extend the capabilities of Document Understanding systems. Finally, AI-based systems promise to be the most powerful Document Understanding tools at the moment.

%This thesis explores the potential of AI-based Document Understanding for automating invoice processing. It will investigate the core technologies, including Optical Character Recognition (OCR), machine learning models for data extraction, and NLP techniques for understanding invoice semantics. By using these new AI technologies, companies can make their accounts payable departments more efficient, save money, improve the accuracy of their financial records, and learn valuable information from their invoices. 

\section{Research Objectives}
%Clearly define the specific goals of the thesis, such as:
%Developing a generative AI model for extracting structured data from financial documents.
%Evaluating the accuracy and efficiency of the model compared to existing methods.
%Exploring the potential applications of the model in real-world financial scenarios.
%Wie effektiv können mit Hilfe von (generativen) Ki Modellen relevante Daten aus semi-strukturierten Finanzdokumenten extrahiert werden am Beispiel der Rechnungsverarbeitung mit dem Document Understanding Modus von UiPath?

%How effectively can AI models extract relevant data from semi-structured financial documents?

%This will be exemplified by invoice processing with UiPath's Document Understanding module.


The research objectives of this thesis include the presentation of the technical foundations of Document Understanding systems and their application. For this, the thesis is structured into a theoretical and an experimental part.

In the theoretical part, the common Document Understanding techniques \ac{OCR}, \ac{IE}, and \ac{NER} are introduced. 
Furthermore, it covers central concepts of \ac{NLP} and \ac{ML} with a focus on \acp{ANN} for Document Understanding. 
Finally, the potential of Document Understanding for the automation of business processes is presented.

After the theoretical foundation is set, this thesis deals with a concrete practical example. The aim is to find evidence for how effectively AI models can extract relevant data from semi-structured financial documents. This will be exemplified by using the UiPath Document Understanding software to extract data from unseen invoice documents.
For this purpose, an invoice processing system is developed using UiPath Document Understanding and tested with a set of actual invoices. Furthermore, the results of the pre-trained out-of-the-box model provided by UiPath are compared to a custom trained model.  

Finally, based on the theoretical and experimental findings it will be concluded whether AI-based Document Understanding Systems can reliably and efficiently automate the processing of invoice documents.
\chapter{Introduction}

\section{Background and Motivation}
% Explain the significance of document understanding in financial accounting. 
% Discuss the limitations of traditional rule-based and machine learning approaches.
% Highlight the potential of generative AI to revolutionize document understanding tasks.

Most content is intended for human readers. However, because of this a computer can only very limitedly process this content. As a consequence, for automation purposes in the first step the given content has to be converted in a format that a machine can read.
\section{Research Objectives}
%Clearly define the specific goals of the thesis, such as:
%Developing a generative AI model for extracting structured data from financial documents.
%Evaluating the accuracy and efficiency of the model compared to existing methods.
%Exploring the potential applications of the model in real-world financial scenarios.
Research Question:  
How effectively can relevant data be extracted from semi-structured financial documents using (generative) AI models, exemplified by invoice processing with UiPath's Document Understanding module?
Wie effektiv können mit Hilfe von (generativen) Ki Modellen relevante Daten aus semi-strukturierten Finanzdokumenten extrahiert werden am Beispiel der Rechnungsverarbeitung mit dem Document Understanding Modus von UiPath?
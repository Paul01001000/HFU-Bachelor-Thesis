\chapter{Theoretical Background}

\section{Document Understanding Techniques}
In general, Document Understanding is the process of ...
Traditionally, Document Understanding has been mainly concerned with the conversion of paper-based documents into a format that is readable for a computer \cite{taylor1994integrated}. With advancing digitization, Document Understanding covers the processing of digitally created documents as well. Paper-based documents start the Document Understanding process as scanned hard-copy of printed documents. Because these scans are essentially images, in the first step the text has to be extracted in a process called \ac{OCR}. While digitally created documents already contain text elements. 
%Review traditional methods like optical character recognition (OCR), named entity recognition (NER), and information extraction (IE).
%Discuss recent advancements in machine learning and deep learning for document understanding.
%Analyze the strengths and weaknesses of existing approaches.
\subsection{Optical Character Recognition}
\ac{OCR}
\subsection{Named Entity Recognition}
\ac{NER}
\subsection{Machine Learning (ML) based techniques}
\ac{ML} is .... \ac{ML}
\section{Generative AI for Natural Language Processing}
\ac{GenAI} is part of \ac{AI}.
\ac{NLP}
%Introduce generative models like generative adversarial networks (GANs) and variational autoencoders (VAEs).
%Explore their applications in natural language processing (NLP) tasks, including text generation, summarization, and translation.
%Discuss the potential benefits of generative AI for document understanding.
\section{Financial Process Automation with Document Understanding}
\subsection{Automated Invoice Processing}
Document Understanding can be combined with \ac{RPA} tools.
\subsection{Alternative Solutions}